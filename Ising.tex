\documentclass{article}
%include package=====================================
\usepackage[utf8]{inputenc}
\usepackage{amsmath}
%Define commend%%%%%%%%%%%%%%%%%%%%%%%%%%%%%%%%%%%%
\newcommand{\ave}[1]{\langle #1\rangle}
\newcommand*\rfrac[2]{{}^{#1}\!/_{#2}}
\newcommand{\si}{{\lbrace s_i\rbrace}}


\title{Ising Model}
\author{Shiling Liang }
\date{February 2017}

\begin{document}

\maketitle


\section{Ising Model}
Ising model is a model used to describe a spin system, in which spins can interacted with each other. A typical ising model is a lattice spins. Those spins on square lattice can only interacted with their nearest neighbors and attempt to align with each other.
To formulate this system, we can write down the energy of the system, which is consisted of two parts, the first part is internal energy, the other is due to the applied field

\begin{equation}
E_\si=E_{in}+E_{out}-\sum_{ij}J_{ij}s_is_j-H\sum_{i=1}^{N}s_i
\end{equation}

where $\si=\{s_1,s_2,\dots,s_N\}$ represent a configuration of the system. J is the coupling constant, which shows how spins interact with each other.In the simplest case, we only consider the nearest neighbor coupling, which meas spins only interacted with their nearest neighbors.
\begin{displaymath}
	J_{ij}=
	\begin{cases}
		J\quad \text{if $i$ and $j$ are nearest neighbors}\\
		0\quad \text{others}
	\end{cases}
\end{displaymath}
For difference spatial construct, the nearest neighbor have different definition.  
\subsection{Partition function}
In statistical physics, the partition function contains all statistical information of a system in thermodynamic equilibrium.
The partition function is defined as
\begin{equation}
\begin{split}
    Z&=\sum_\si\exp{-\beta E_\si}\\
    &=\sum_\si\exp{(-\sum_{ij}J_{ij}s_is_j-H\sum_{i=1}^{N}s_i)}
\end{split}
\end{equation}

and let $p_\si$ denote the probability of the system being in a particular state $\si$, which is
\begin{equation}
p_\si=\frac{\exp{-\beta E_\si}}{Z}
\end{equation}
Then the esemble average of any observable $A$ can be expressed as
\begin{equation}
    \begin{split}
        \ave{A}=A_\si p_\si
    \end{split}
\end{equation}

\section{Simulation}
\subsection{Quantity of interest}
\paragraph{Magnetization}
In intuition, total magnetization is a important quantity of a spin system. It is just the direct summation of the total spins. While, to make it more comparable for systems of different sizes, we use magnetization per spin instead. Which is defined as
\begin{equation}
M=\frac{1}{N}\sum_i{\sigma_i}
\end{equation}
for a statistical system, we care more about its average value, which can be derived from partition function function. 
\begin{equation}
\end{equation}
\paragraph{Correlation length}
In high temperature, the spins point up and down randomly. While in low temperature, all spines tend to align to each other. To quantify this, we can define a quantity named correlation length $\xi$, which can be defined mathematically through correlation function $g(i,j)$
\begin{equation}
g(i,j)=\langle \sigma_i-\langle\sigma_i\rangle\rangle \langle \sigma_j-\langle\sigma_j\rangle\rangle
\end{equation}
which measured the largest cluster of aligned spins.
\section{Algorithms}
\subsection{Exact calculation}
For a finite system, in principle, we can list all possible configurations. So that we can calculate all thermal quantities of this system, i.e. the analytical solution of the system. While, for large system, the time computational complexity is $O(2^n)$. Hence, this algorithm is not suitable for large systems.

For a $4\times 4$ spin system, the number of all possible configurations is $N=2^16=65536$, which is still not a large number so that we can use enumeration algorithm to calculate it.


\subsection{Monte Carlo algorithms}

\subsubsection{Metropolis Algorithm}
Metropolis algorithm is a general algorithm, there are some different ways to implement this algorithm. 

As we have obtained exact result of a 4-by-4 system, we then applied Metropolis algorithm on a 4-by-4 system and compare it with the exact calculation.


\subsubsection{Wolff Cluster Algorithm}
\paragraph{Critical slowing down}
At the vicinity of critical point, this algorithm will take a very long time to reach Boltzmann distribution. When
temperature come closely to critical point, the correlation length increase rapidly.For a infinite size system, the correlation length even divergent at critical temperature, while for a finite system, it can exceed the length of the system.  Many clusters would occur in the system in which the spins aligned to each other. So that it's hard to flip spins in these cluster. This kind of phenomenon is so-called critical slowing down. To over come it, we can use Wolff cluster algorithm, which flip clusters rather than single every time.
\paragraph{implementation}
Before flipping a cluster, we need to construct clusters.

\subsubsection{Heat Bath Algorithm}
\end{document}
